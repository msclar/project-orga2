\documentclass[a4paper]{article}

\usepackage[paper=a4paper, left=1.5cm, right=1.5cm, bottom=1.5cm, top=3.5cm]{geometry}
\usepackage[spanish,activeacute]{babel}
\usepackage[utf8]{inputenc}
\usepackage{amsthm}
\usepackage{amsmath}
\usepackage{amsfonts}
\usepackage{amssymb}
\usepackage{alltt}
\usepackage{graphicx} %Para incluir el logo de la UBA
\usepackage{caratula} %Para armar el cuadro de integrantes
\usepackage{multirow} %Para poder poner varias lineas juntas sin divisiones en una tabla
\usepackage[lined,ruled,linesnumbered]{algorithm2e}
\usepackage{algpseudocode}
\usepackage{scrextend}
\usepackage{blindtext}


%Cosas para escribir codigo fuente
%Fuente: http://en.wikibooks.org/wiki/LaTeX/Source_Code_Listings
\usepackage{listings}
\usepackage{color}

\setcounter{secnumdepth}{5}

\definecolor{mygreen}{rgb}{0,0.6,0}
\definecolor{mygray}{rgb}{0.5,0.5,0.5}
\definecolor{myorange}{rgb}{1,0.4,0.2}
\definecolor{myblue}{rgb}{0,0,0.65}

%Configuracion para los listings
\lstset{ %
  backgroundcolor=\color{white},   % choose the background color; you must add \usepackage{color} or \usepackage{xcolor}
  basicstyle=\small,        % the size of the fonts that are used for the code
  breakatwhitespace=false,         % sets if automatic breaks should only happen at whitespace
  breaklines=true,                 % sets automatic line breaking
  captionpos=b,                    % sets the caption-position to bottom
  commentstyle=\color{mygreen},    % comment style
  deletekeywords={...},            % if you want to delete keywords from the given language
  escapeinside={\%*}{*)},          % if you want to add LaTeX within your code
  extendedchars=true,              % lets you use non-ASCII characters; for 8-bits encodings only, does not work with UTF-8
  frame=single,                    % adds a frame around the code
  keywordstyle=\color{myblue},       % keyword style
  language=Octave,                 % the language of the code
  morekeywords={*,...},            % if you want to add more keywords to the set
  numbers=left,                    % where to put the line-numbers; possible values are (none, left, right)
  numbersep=5pt,                   % how far the line-numbers are from the code
  numberstyle=\tiny\color{mygray}, % the style that is used for the line-numbers
  rulecolor=\color{black},         % if not set, the frame-color may be changed on line-breaks within not-black text (e.g. comments (green here))
  showspaces=false,                % show spaces everywhere adding particular underscores; it overrides 'showstringspaces'
  showstringspaces=false,          % underline spaces within strings only
  showtabs=false,                  % show tabs within strings adding particular underscores
  stepnumber=1,                    % the step between two line-numbers. If it's 1, each line will be numbered
  stringstyle=\color{myorange},     % string literal style
  tabsize=2,                       % sets default tabsize to 2 spaces
  title=\lstname                   % show the filename of files included with \lstinputlisting; also try caption instead of title
}

\renewcommand{\lstlistingname}{C\'{o}digo}

\lstset{language=C++,caption={Descriptive Caption Text},label=DescriptiveLabel}
\setlength\parindent{0pt}

%\topmargin = -1cm
%\textheight = 24cm 

\begin{document}


\integrante{Sclar, Melanie}{551/12}{melaniesclar@gmail.com}
\integrante{Zylber, Ariel}{530/12}{arielzylber@gmail.com}

\def\Materia{Final de Organizaci\'on del Computador II}
\def\Fecha{Alg\'un d\'ia}

%----- CARATULA -----%

\thispagestyle{empty}

\begin{center}
	\includegraphics[scale = 0.25]{logo_uba.jpg}
\end{center}

\vspace{5mm}

\begin{center}
	{\textbf{\large UNIVERSIDAD DE BUENOS AIRES}}\\[1.5em]
	{\textbf{\large Departamento de Computaci\'{o}n}}\\[1.5em]
    {\textbf{\large Facultad de Ciencias Exactas y Naturales}}\\
    \vspace{35mm}
    {\LARGE\textbf{\Materia}}\\[1em]    
    \vspace{15mm}
    {\Large \textbf{\Titulo}}\\[1em]
    \vspace{15mm}
    {\textbf{\Large \Fecha}}\\
    \vspace{15mm}
    \textbf{\tablaints}
\end{center}

\newtheorem{teo}{Teorema}[section]
\newtheorem{propo}{Proposici\'{o}n}[section]
\newtheorem{lema}{Lema}[section]
\newtheorem{coro}{Corolario}[section]
\newtheorem{defi}{Definici\'{o}n}[section]

\newpage
\setcounter{page}{1}
\pagenumbering{arabic}
\pagestyle{plain}

\newpage


\newcommand{\Asig}{\ensuremath{\leftarrow}}
\newcommand{\AndY}{\ensuremath{\wedge}}
\newcommand{\Or}{\ensuremath{\vee}}
\newcommand{\Not}{\ensuremath{\neg}}
\newcommand{\NotEq}{\ensuremath{\neq}}
\newcommand{\MayorIg}{\ensuremath{\geq}}
\newcommand{\tabu}{\hspace*{0.7cm}}
\newcommand{\ctabu}{\hspace*{0.8cm}}
\newcommand{\htabu}{\hspace*{0.35cm}}
\newcommand{\moduloNombre}[1]{\textbf{#1}}

\section{Objetivo del trabajo}

El objetivo del trabajo es resolver la ecuación de Biocalor de Pennes utilizando el método de Jacobi para calcular la temperatura en cada sección de un tejido a cada instante de tiempo. Esto se enmarca dentro de los proyectos del Laboratorio de Sistemas Complejos (DC-UBA), pues el LSC aplicará nuestro trabajo para agilizar el cálculo de una sección de su modelo para identificar la temperatura y posición óptimas en las que aplicar un tratamiento para reducir los tumores cancerígenos. A grandes rasgos, los tratamientos ultilizados consiste en colocar un ánodo y un cátodo en puntos específicos de la piel del paciente cercanos al tumor y lograr la destrucción del tejido cancerígeno debido a cambios de pH, aumento de permeabilidad irreversible en la membrana plasmática o aumento en la absorción del quimioterápico, entre otros. Como efecto secundario del tratamiento se produce un aumento de temperatura en el tejido. Nosotros desarrollaremos un modelo para simular calentamiento del tejido y así poder estudiar que consecuencias trae al paciente. Gráficamente,

%% insertar dibujo de emmanuel

Como la resolución de esta ecuación ocupa una parte no despreciable del cómputo de sus algoritmos completos, lo implementaremos en C plano, en ASM sin SIMD y con SIMD y compararemos los tiempos de ejecución obtenidos con cada una (compilando con gcc e icc). Creemos que utilizar SIMD podría ser muy provechoso en este problema, ya que muchas de las operaciones de matrices que efectuaremos son paralelizables y esperamos ver una significante reducción en el tiempo de ejecución del algoritmo.

\section{Distribuci\'on de la temperatura: obtención del sistema a resolver}

\subsection{Temperatura en todo el tejido salvo los bordes}
Partimos de la ecuación de Biocalor de Pennes, 
$$\nabla \cdot (k \nabla T) - w_b C_b \rho_b (T-T_a) + q''' + \sigma |\nabla \varphi|^2 = \rho \ C_\rho \ \frac{\partial T}{\partial t}$$

Donde $\sigma$ es la conductividad eléctrica, $\phi$ es el potencial eléctrico, $k$ es la conductividad térmica, $T_a$ es la temperatura arterial, $w_b$ es la blood perfusion rate, $C_b$ es la capacidad calorífica de la sangre, $\rho_b$ es la densidad de la sangre, $q'''$ es la generación de calor metabólico, $\rho$ es la densidad del tejido y $C_\rho$ es la capacidad calorífica del tejido.\\

Para nuestro modelo, hay varias constantes que suponemos conocidas, en los siguientes valores:

$C_\rho= 3680$, $\rho= 1039$, $q'''= 10437$, $C_b= 3840$, $\rho_b= 1060$, $w_b=7.15 \ 10^{-3}$, $T_a = 310.15^\circ K$ \\

$k$, $\sigma$ y $\varphi$ son campos escalares variables. Sin embargo, a la hora de realizar los cálculos tomaremos los siguientes valores para las funciones,
como fue indicado por Emmanuel. 

$k(x,y)=0.565$ \\
$\sigma(x,y)=0.75$ \\

Para calcular $\varphi$ supondremos dado un ánodo en la posición $(x_a, y_a)$ con potencial eléctrico constante $q_a$ y un cátodo en la posición $(x_c, y_c)$ con potencial eléctrico constante $q_c$. Luego obtenemos como se extiende la $\varphi$ a todo el recinto resolviendo la ecuación de Laplace dados los puntos del ánodo y cátodo fijos y condición de borde de flujo cero. Para más detalles sobre como calcular esto ver apéndice B.\\

Notar que en estos valores particulares los campos son constantes en el tiempo pero al realizar la discretización del problema no lo supondremos constante que sino lo haremos en la versión general donde sus valores podrían depender del tiempo.

Entonces, los únicos valores desconocidos para nosotros son los $T_n(i,j)$, es decir, el valor de la temperatura en el tiempo $n$ (ya discretizado) en cada lugar del espacio $(i,j)$.
Utilizaremos las discretizaciones tal y como nos fueron indicadas: el término temporal por diferencias adelantadas, los términos espaciales por diferencias centradas. \\

A continuación calcularemos el término $\nabla \cdot (k \nabla T)$:
\begin{equation} \label{eq:nablaT}
\begin{split}
\nabla \cdot (k \nabla T) & = \nabla \cdot k \bigg(\frac{\partial T}{\partial x}, \frac{\partial T}{\partial y}\bigg) \\
& = \frac{\partial}{\partial x} \bigg(k \frac{\partial T}{\partial x}\bigg) + \frac{\partial}{\partial y} \bigg(k \frac{\partial T}{\partial y}\bigg) \\
& = \frac{\partial k}{\partial x} \cdot \frac{\partial T}{\partial x} + k \cdot \frac{\partial^2 T}{\partial x^2} + \frac{\partial k}{\partial y} \cdot \frac{\partial T}{\partial y} + k \cdot \frac{\partial^2 T}{\partial y^2} \\
 & = \frac{\partial k}{\partial x} \cdot \frac{\partial T}{\partial x} + \frac{\partial k}{\partial y} \cdot \frac{\partial T}{\partial y} + k \cdot \frac{\partial^2 T}{\partial x^2} + k \cdot \frac{\partial^2 T}{\partial y^2}
\end{split}
\end{equation}

Notar que si $k$ es constante, $\frac{\partial k}{\partial x} = \frac{\partial k}{\partial y} = \frac{\partial k}{\partial t} = 0$ y por lo tanto, obtenemos $\nabla \cdot (k \nabla T) = k \cdot \frac{\partial^2 T}{\partial x^2} + k \cdot \frac{\partial^2 T}{\partial y^2} + k \cdot \frac{\partial^2 T}{\partial t^2}$ \\
Sin embargo, a priori no sabemos que nuestro modelo de aplicación será con $k$ constante así que deberemos calcular sus derivadas. Mostramos cómo se calculan todas las derivadas respecto de $T$ con sus discretizaciones, y para $k$ será análogo.

\begin{equation*}
\begin{split}
\frac{\partial T}{\partial x} & \approx \frac{T_{n+1}(i+1,j) - T_{n+1}(i-1,j)}{\Delta x_i + \Delta x_{i+1}} \\
\frac{\partial T}{\partial y} & \approx \frac{T_{n+1}(i,j+1) - T_{n+1}(i,j-1)}{\Delta y_j + \Delta y_{j+1}} \\
\frac{\partial^2 T}{\partial x^2} & \approx \frac{2 T_{n+1}(i+1,j)}{\Delta x_i (\Delta x_i + \Delta x_{i-1})} - \frac{2 T_{n+1}(i,j)}{\Delta x_i \Delta x_{i-1}} + \frac{2 T_{n+1}(i-1,j)}{\Delta x_{i-1} (\Delta x_i + \Delta x_{i-1})} \\
\frac{\partial^2 T}{\partial y^2} & \approx \frac{2 T_{n+1}(i,j+1)}{\Delta y_j (\Delta y_j + \Delta y_{j-1})} - \frac{2 T_{n+1}(i,j)}{\Delta y_j \Delta y_{j-1}} + \frac{2 T_{n+1}(i,j-1)}{\Delta y_{j-1} (\Delta y_j + \Delta y_{j-1})} \\
\frac{\partial T}{\partial t} & \approx \frac{T_{n+1}(i,j) - T_n(i,j)}{\Delta t_n}
\end{split}
\end{equation*}

Como podemos asumir que $\Delta x_i = \Delta x_j = \Delta x$ y $\Delta y_i = \Delta y_j = \Delta y$ para todo $i$, $j$, las ecuaciones quedan así:
\begin{equation} \label{eq:derivadas}
\begin{split}
\frac{\partial T}{\partial x} & \approx \frac{T_{n+1}(i+1,j) - T_{n+1}(i-1,j)}{2\Delta x}  \\
\frac{\partial T}{\partial y} & \approx \frac{T_{n+1}(i,j+1) - T_{n+1}(i,j-1)}{2\Delta y} \\
\frac{\partial^2 T}{\partial x^2} & \approx \frac{T_{n+1}(i+1,j)}{(\Delta x)^2} - \frac{2 T_{n+1}(i,j)}{(\Delta x)^2} + \frac{T_{n+1}(i-1,j)}{(\Delta x)^2} \\
\frac{\partial^2 T}{\partial y^2} & \approx \frac{T_{n+1}(i,j+1)}{(\Delta y)^2} - \frac{2 T_{n+1}(i,j)}{(\Delta y)^2} + \frac{T_{n+1}(i,j-1)}{(\Delta y)^2} \\
\frac{\partial T}{\partial t} & \approx \frac{T_{n+1}(i,j) - T_n(i,j)}{\Delta t}
\end{split}
\end{equation}

Aplicando todo esto a la ecuación del Biocalor de Pennes, despejamos los coeficientes de $T_n$ en función de los de $T_{n+1}$, para obtener un sistema de ecuaciones a resolver con Jacobi. Tomemos la ecuación del Biocalor de Pennes:
$$\nabla \cdot (k \nabla T) - w_b C_b \rho_b (T-T_a) + q''' + \sigma |\nabla \varphi|^2 = \rho \ C_\rho \ \frac{\partial T}{\partial t}$$

Reemplazando con lo obtenido en (\ref{eq:nablaT}),
$$ \frac{\partial k}{\partial x} \cdot \frac{\partial T}{\partial x} + \frac{\partial k}{\partial y} \cdot \frac{\partial T}{\partial y} + k \cdot \frac{\partial^2 T}{\partial x^2} + k \cdot \frac{\partial^2 T}{\partial y^2} - w_b C_b \rho_b (T-T_a) + q''' + \sigma |\nabla \varphi|^2 = \rho \ C_\rho \ \frac{\partial T}{\partial t}$$

Luego, aproximamos las derivadas como definimos en (\ref{eq:derivadas}):\\
\begin{equation*}
\begin{aligned}
\rho \ C_\rho \ \frac{T_{n+1}(i,j) - T_n(i,j)}{\Delta t} &= \frac{\partial k}{\partial x} \cdot \frac{T_{n+1}(i+1,j) - T_{n+1}(i-1,j)}{2\Delta x} \\ & \ \ \ \ \ \ \  + \frac{\partial k}{\partial y} \cdot \frac{T_{n+1}(i,j+1) - T_{n+1}(i,j-1)}{2\Delta y} \\ & \ \ \ \ \ \ \ + k \cdot \bigg(\frac{T_{n+1}(i+1,j)}{(\Delta x)^2} - \frac{2 T_{n+1}(i,j)}{(\Delta x)^2} + \frac{T_{n+1}(i-1,j)}{(\Delta x)^2}\bigg) \\ & \ \ \ \ \ \ \  + k \cdot \bigg(\frac{T_{n+1}(i,j+1)}{(\Delta y)^2} - \frac{2 T_{n+1}(i,j)}{(\Delta y)^2} + \frac{T_{n+1}(i,j-1)}{(\Delta y)^2}\bigg) \\ & \ \ \ \ \ \ \  -  w_b C_b \rho_b (T_{n+1}(i,j)-T_a) + q''' + \sigma |\nabla \varphi|^2\\
\end{aligned}
\end{equation*}

\bigskip
Ahora sí, despejemos $T_n$ en función de los de $T_{n+1}$. Para abreviar, sea $TInd = w_b C_b \rho_b T_a + q''' + \sigma |\nabla \varphi|^2$. \\

\begin{equation*}
\begin{aligned}
- T_n(i,j) \frac{\rho \ C_\rho}{\Delta t} - TInd(i,j) &= T_{n+1}(i+1,j) \cdot \bigg(\frac{\partial k}{\partial x} \cdot \frac{1}{2\Delta x} +  k \cdot\frac{1}{(\Delta x)^2}\bigg) \\ & \ \ \ \ \ \ \ + T_{n+1}(i-1,j) \cdot \bigg(-\frac{\partial k}{\partial x} \cdot \frac{1}{2\Delta x} + k \cdot\frac{1}{(\Delta x)^2}\bigg) \\ & \ \ \ \ \ \ \ + T_{n+1}(i,j) \cdot \bigg(-\frac{2k}{(\Delta x)^2} -\frac{2k}{(\Delta y)^2} - w_b C_b \rho_b - \rho \ C_\rho \ \frac{1}{\Delta t}\bigg) \\ & \ \ \ \ \ \ \ + T_{n+1}(i,j+1) \cdot \bigg(\frac{\partial k}{\partial y} \cdot \frac{1}{2\Delta y} + k \cdot\frac{1}{(\Delta y)^2}\bigg) \\ & \ \ \ \ \ \ \  + T_{n+1}(i,j-1) \cdot \bigg(-\frac{\partial k}{\partial y} \cdot \frac{1}{2\Delta y} + k \cdot\frac{1}{(\Delta y)^2}\bigg)
\end{aligned}
\end{equation*}

Con estas ecuaciones nos quedará un sistema a resolver, pero antes de plantearlo debemos explicitar cuáles serán las condiciones de borde en este problema, ya que las ecuaciones antes dichas no valen en el borde exterior ni alrededor del ánodo y cátodo.

\subsection{Temperatura en los bordes}
La condición de borde del sistema es de flujo cero, es decir $q \cdot \hat{n} = 0$ donde $\hat{n}$ es el vector unitario normal al borde del recinto. Esto quiere decir que $-K \nabla T \cdot \hat{n} = 0$, dividiendo por $-K$ queda $\nabla T \cdot \hat{n} = 0$. Esto se interpreta como que en el borde la derivada en la dirección hacia adentro es nula. Al discretizar, esto se traduce en que la diferencia entre la temperatura en un punto del borde y su inmediato vecino interno es nula, es decir que son iguales. Más formalmente, asumiremos que:
\begin{equation*}
\begin{aligned}
T_{n+1}(0,j) & = T_{n+1}(1,j) \\ 
T_{n+1}(i,0) & = T_{n+1}(i,1) \\
T_{n+1}(i,max_j) & = T_{n+1}(i,max_j-1) \\ 
T_{n+1}(max_i,j) & = T_{n+1}(max_i-1,j)
\end{aligned}
\end{equation*}

Además, en el borde de los electrodos (ánodo y cátodo) el cálculo de la temperatura es diferente. Los detalles de su cálculo se pueden ver en el Apéndice A. Siendo $(i,j)$ la aproximación del borde del electrodo,

$$T_{n+1}(i,j) = \frac{r}{4(r-1)} \cdot \big( T_{n+1}(i+1,j) + T_{n+1}(i,j+1) + T_{n+1}(i-1,j) + T_{n+1}(i,j-1)\big) - \frac{T_{aire}}{r-1}$$

Donde $r = \frac{k}{\Delta x \cdot h}$, $h = 10 \frac{W}{m^2 K}$ y $T_{aire} = 296 ^\circ K$.\\ 

En nuestro trabajo, las posiciones de los electrodos serán parte del input del programa.

\subsection{Planteo del sistema y resolución}

Habiendo hallado todas las ecuaciones que relacionan $T_n$ con $T_{n+1}$, podemos escribir un sistema a resolver de la forma $$A \cdot T_{n+1} = -\frac{\rho \cdot C_\rho}{\Delta t} \cdot T_n - TInd$$ y despejar $T_{n+1}$ utilizando Jacobi. Consideramos a $A$ como la matriz cuadrada de $max_i*max_j$ donde cada fila (y cada columna) corresponde a una posición $(i, j)$ de la discretización de modo que en la fila de la posición $(i, j)$ tendremos los coeficientes de la ecuación correspondiente a $T_n(i,j)$ para cada posición de la grilla. Notemos que luego de hallar $T_n$, la única incógnita del sistema es en efecto $T_{n+1}$ y en efecto se puede aplicar Jacobi usualmente.\\

Notemos que la matriz $A$ es esparsa, pues en cada fila los coeficientes que podrían ser no nulos corresponden a las posiciones del punto en cuestión y sus vecinos. Con esto conseguiremos reducir mucho el tiempo de cómputo en todas las implementaciones.

\newpage
\section{Apéndice A: cálculo de temperatura en los electrodos}

Vamos a calcularlo para la pared vertical derecha de cada electrodo, los otros cálculos son análogos. El electrodo tiene 4 \textit{paredes}, pues al discretizar el tejido en una grilla el punto donde se apoya el electrodo quedará dentro de una de las casillas de la grilla.

Siendo $k$ la función definida anteriormente, sea $h = 10 \frac{W}{m^2 K}$ y $T_{aire} = 296 ^\circ K$

\begin{equation*}
\begin{aligned}
-k \frac{\partial T}{\partial x} & = h (T_{n+1}(i,j) -  T_{aire}) \\
-k \ \frac{T_{n+1}(i+1,j) - T_{n+1}(i,j)}{\Delta x} & \approx h (T_{n+1}(i,j) -  T_{aire}) \\
\frac{k}{\Delta x \cdot h} \cdot (T_{n+1}(i,j) - T_{n+1}(i+1,j)) & \approx T_{n+1}(i,j) -  T_{aire}
\end{aligned}
\end{equation*}

Utilizando el renombre $r = \frac{k}{\Delta x \cdot h}$, nos queda:

\begin{equation*}
\begin{aligned}
r \cdot (T_{n+1}(i,j) - T_{n+1}(i+1,j)) & \approx T_{n+1}(i,j) - T_{aire} \\
r \cdot T_{n+1}(i,j) - r \cdot T_{n+1}(i+1,j) & \approx T_{n+1}(i,j) - T_{aire} \\
r \cdot T_{n+1}(i,j) - T_{n+1}(i,j) & \approx r \cdot T_{n+1}(i+1,j) - T_{aire} \\
(r - 1) \cdot T_{n+1}(i,j) & \approx r \cdot T_{n+1}(i+1,j) - T_{aire} \\
T_{n+1}(i,j) & \approx \frac{r \cdot T_{n+1}(i+1,j) - T_{aire}}{r-1}
\end{aligned}
\end{equation*}

Esto mismo se puede calcular para las otras tres paredes del electrodo. En resumen, los cuatro cálculos darán como resultado lo siguiente:

\begin{equation*} \label{eq:paredes}
\begin{aligned}
T_{n+1}(i,j) & \approx \frac{r \cdot T_{n+1}(i+1,j) - T_{aire}}{r-1} \\
T_{n+1}(i,j) & \approx \frac{r \cdot T_{n+1}(i-1,j) - T_{aire}}{r-1} \\
T_{n+1}(i,j) & \approx \frac{r \cdot T_{n+1}(i,j+1) - T_{aire}}{r-1} \\
T_{n+1}(i,j) & \approx \frac{r \cdot T_{n+1}(i,j-1) - T_{aire}}{r-1}
\end{aligned}
\end{equation*}

Finalmente, aproximamos el valor de la temperatura de la casilla donde se coloca el electrodo como el promedio de la temperatura en sus cuadro bordes. Sumando las cuatro ecuaciones de (\ref{eq:paredes}):

\begin{equation*}
\begin{aligned}
4 \cdot T_{n+1}(i,j) & \approx \frac{r \cdot [T_{n+1}(i+1,j) + T_{n+1}(i-1,j) + T_{n+1}(i,j+1) + T_{n+1}(i,j-1)] - 4 \cdot T_{aire}}{r-1} \\
T_{n+1}(i,j) & \approx \frac{r}{4 \cdot (r-1)} \cdot \big(T_{n+1}(i+1,j) + T_{n+1}(i-1,j) + T_{n+1}(i,j+1) + T_{n+1}(i,j-1)\big) - \frac{T_{aire}}{r-1}
\end{aligned}
\end{equation*}

\newpage
\section{Apéndice B: cálculo del potencial eléctrico}

El potencial eléctrico es un campo escalar definido como la solución de la ecuación de Laplace ($\delta \varphi = 0$) dados dos puntos fijos (ánodo y ćatodo) y condición de borde de flujo cero.\\
La discretización de la ecuación $\delta \varphi = 0$ es conocida y corresponde a que una posición es el promedio de sus vecinos. Es decir, $$\varphi(i, j) = \frac{\varphi(i-1, j) + \varphi(i+1, j) + \varphi(i, j-1) + \varphi(i, j+1)}{4}$$
Por otro lado, la condición de borde es la misma que en el caso de la ecuación de calor, por lo que las fórmulas de las discretizaciones en los puntos del borde son análogas y equivalen a:
\begin{equation*}
\begin{aligned}
\varphi(0,j) & = \varphi(1,j) \\ 
\varphi(i,0) & = \varphi(i,1) \\
\varphi(i,max_j) & = \varphi(i,max_j-1) \\ 
\varphi(max_i,j) & = \varphi(max_i-1,j)
\end{aligned}
\end{equation*}
Para obtener una solución aproximada del sistema, iteramos reemplazando el potencial actual de cada posición por el promedio de sus vecinos hasta que estos valores converjan.
\end{document}
